\documentclass{article}
\title{Billiards}
\begin{document}
\maketitle

%add a chapter
\section{The rules of billiard dynamics}
\paragraph{Consider a simple polygonal billiard table, and let a projectile be free to move inside it, similar to a cue ball (Fig 1.1). We assume that all collisions with the sides of the square are completely elastic (i.e., angle of incidence = angle of reflection, and no loss of speed occurs after the collision). We further assume that any collision with a vertex will prevent any further motion (one can imagine that the projectile gets “stuck” at the vertex).}

%Fig 1.1

\paragraph{Note- For reasons that will be clear later, we restrict our discussion to tables in the shape of polygons with an even no of sides, with opposite sides being parallel.}
 
\paragraph{With these rules, we may form possible trajectories inside the table. And these trajectories may be studied using various geometrical manipulations. One powerful method that will be the backbone of our analysis is the idea of identifying edges of a polygon- Given a table of a particular polygon shape, each side can be joined (or “identified”) to its opposite parallel side to create a curved surface (Fig 1.2). This is useful, because it allows us to convert any trajectory into a straight line on that curved surface (Fig 1.3)[DJ3] , allowing us to use the geometrical properties of these surfaces to extract information about the trajectories. Note that in the 2D representation of an identified polygon surface, a projectile approaching a side behaves as if it goes into that side, and comes out its identified side (rather like a Pac-man screen)}

\paragraph{Note- It is clear now why we restrict ourselves to polygonal tables with even number of sides- only these shapes can be identified to form a surface.}

%Fig 1.2
%Fig 1.3


\end{document}

