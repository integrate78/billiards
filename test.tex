\documentclass{article}
\title{Billiards}
\usepackage{amsmath}
\begin{document}
\maketitle

%add a chapter
\section{The rules of billiard dynamics}
\paragraph{Consider a simple polygonal billiard table, and let a projectile be free to move inside it, similar to a cue ball (Fig 1.1). We assume that all collisions with the sides of the square are completely elastic (i.e., angle of incidence = angle of reflection, and no loss of speed occurs after the collision). We further assume that any collision with a vertex will prevent any further motion (one can imagine that the projectile gets “stuck” at the vertex).}

%Fig 1.1

\paragraph{Note- For reasons that will be clear later, we restrict our discussion to tables in the shape of polygons with an even no of sides, with opposite sides being parallel.}
 
\paragraph{With these rules, we may form possible trajectories inside the table. And these trajectories may be studied using various geometrical manipulations. One powerful method that will be the backbone of our analysis is the idea of identifying edges of a polygon- Given a table of a particular polygon shape, each side can be joined (or “identified”) to its opposite parallel side to create a curved surface (Fig 1.2). This is useful, because it allows us to convert any trajectory into a straight line on that curved surface (Fig 1.3)[DJ3] , allowing us to use the geometrical properties of these surfaces to extract information about the trajectories. Note that in the 2D representation of an identified polygon surface, a projectile approaching a side behaves as if it goes into that side, and comes out its identified side (rather like a Pac-man screen)}

\paragraph{Note- It is clear now why we restrict ourselves to polygonal tables with even number of sides- only these shapes can be identified to form a surface.}

%Fig 1.2
%Fig 1.3

\section{Basic terminology and characterization of trajectories}
\paragraph{Identifying a polygon into a surface results, as one may note, into a continuous representation of the trajectory with a single constant slope at all points. Thus, a trajectory can be uniquely identified by this slope. We will refer to it as \textit{m}}

\paragraph{Given that a projectile moves with slope m then, we would like to be able to keep track of its motion within the polygon. Thus, a method of “bookkeeping” of the projectile’s motion is required to characterize trajectories. One way to do so is by keeping track of which sides the projectile goes through. This is made easy by side identification, since it allows us to treat two identified sides as essentially the same entity.}

\paragraph{We can thus label identified sides uniquely, and create a sequence of all the identified sides that the projectile goes through (Fig 1.4 ). If we write this sequence like a word, then we attain what is known as a cutting sequence. Fig 4 shows some examples}

%Fig 1.4

\paragraph{Describing a trajectory as a cutting sequence, we may try to ask some explicit questions about the trajectory, such as}

\begin{itemize}
\item Given a polygon (and hence the number of labelling letters), which sequences of these letters form a valid cutting sequence?

\item Given a valid cutting sequence, is there any way to use them to determine which trajectories are periodic and which are not?

\item Do cutting sequences uniquely represent a single trajectory (does a unique mapping exist between m and a cutting sequence?), or a group of trajectories? And if so, what kind of precise relationship exists between cutting sequences and trajectories?

\end{itemize}

\paragraph{With these basic rules and tools, we now delve into an analysis of these polygon tables geometrically}

\section{The Euler Characteristic $\chi$}

\paragraph{Knowledge of the kind of surface we obtain after identifying a polygon can tell us how to approach the analysis of billiard dynamics on it. We may try to predict the obtained surface by tricky geometrical manipulations and visualization, or we may use the concept of the Euler Characteristic.}

\paragraph{We define Euler Characteristic as follows}

\paragraph{\textit{DEFINITION:} Given a surface \textit{S} made by identifying edges of polygons, with \textit{V} vertices, \textit{E} edges, and \textit{F} faces, its Euler characteristic $\chi$ is}

\begin{equation}
\mathit{\chi(S)=V-E+F}
\end{equation}

\paragraph{In topology, the genus (the number of “holes” in a surface) of the surface plays an important role. From classification of the surfaces to developing homeomorphism between two surfaces, we need to know the genus of the surface.}

\paragraph{The following theorem is an explicit relationship between Euler’s characteristic and the genus of an orientable surface (any surface with a well defined orientation):}

\paragraph{\textit{THEOREM}: A surface \textit{S} with genus \textit{g} has Euler characteristic given by}

\begin{equation}
\chi=2-2g
\end{equation}

\paragraph{We use these numbers to identify a surface uniquely. If two polygons can both be identified into a common surface with a particular Euler Characteristic $\chi$, then the dynamics for both polygons are the same (since in the surface, the dynamics end up being the same). Since the Euler characteristic can be calculated from just knowledge of the polygon, we can figure out which surface a polygon can form without any complex visualizations.}

\paragraph{In order to apply the Euler characteristic to our use, we must know vertices, edges and faces of the polygon. For a given polygon with identified edges, we can find edges and faces of the polygon without much effort.}

\paragraph{To count the vertices, we implement the following sequence of steps:}


\begin{enumerate}

\item Pick a side 1, it has an identified side 1’. Pick a vertex on 1. It has an identified vertex on 1’.

\item The chosen vertex on side 1 is connected to another side (call it 2. The side 2’, identified to 2 is connected to 1’ in the same sense (clockwise or anti-clockwise) as 2 is to 1.

\item Since 2 has the chosen vertex, the corresponding vertex on 2’ can be identified to the two vertices already identified in step 1.

\item Continue this process till we reach the original polygon vertex from step 1. At this point, choose any unidentified vertices, and repeat the process.

\end{enumerate}


\paragraph{We iterate this process repeatedly till we have covered every polygon vertex (Note that the number of iterations = the number of surface vertices)}

\paragraph{Figure (1.7) illustrates vertex counting scenario for a hexagon}

%Fig 1.7

\paragraph{A number of theorems can be proven regarding the number of vertices in a polygon identified surface. We illustrate a few of them in the appendix}


\section{Symmetries and transformations}

\paragraph{Symmetries of a polygon are linear bijective transformations which map between points in the same space such that vertices of a polygon are mapped back to vertices (a vertex may be mapped to a different one). 
These symmetries are of interest because, among other reasons, a symmetric transformation of the polygon, applied to a valid trajectory on that polygon, gives us another valid trajectory.}

\begin{enumerate}

\item \textbf{Reflection along x} 
$ (\begin{bmatrix}
1&0\\0&-1
\end{bmatrix})$
\textbf{or y}
$ (\begin{bmatrix}
-1&0\\0&1
\end{bmatrix})$

\item \textbf{Rotation by $\pi/2$ radians}
$ (\begin{bmatrix}
0&-1\\1&0
\end{bmatrix})$

\item \textbf{Shearing}

%Fig 1.6
%Fig 1.7

\end{enumerate}

\paragraph{Among these, shearing requires a bit more careful study.}



\section{Shears and one directional scaling}


\paragraph{When a surface is twisted in a direction such that points on a line parallel to this direction remain stationary, we call it shearing of the surface. (Fig 1.8)}

%Fig 1.8

\paragraph{\textit{DEFINITION}: A linear transformation of the form
$ (\begin{bmatrix}
1&m\\0&1
\end{bmatrix})$
or
$ (\begin{bmatrix}
1&0\\m&1
\end{bmatrix})$
where m is any integer, is a one directional shear map.
}

\paragraph{There are two types of shear maps:}

\begin{itemize}
\item \textbf{Horizontal shear mapping takes a point on the surface (x,y) and transforms it to (x+my,y), where m is a fixed constant known as shear factor. Such a shear is also known as shear parallel to x-axis. It is represented by 
$ (\begin{bmatrix}
1&m\\0&1
\end{bmatrix})$ }

\item \textbf{Vertical shear mapping takes a point on the surface (x,y) and transforms it to (x,mx+y), where m is a fixed constant known as shear factor. Such a shear is also known as shear parallel to y-axis. It is represented by
$ (\begin{bmatrix}
1&0\\m&1
\end{bmatrix})$ }
\end{itemize}

\paragraph{We must not confuse shearing with rotation. Shearing distorts the 2D shape of the surface. If we rotate a square, we will always get a square in 2D.}

\paragraph{Note that shearing preserves properties like parallelism and area. It also preserves the side identifications of the polygon (since parallel sides remain parallel to each other after shearing). Thus, shearing a polygon does not change the 3D surface it is identified into.}

\paragraph{We will usually cut the sheared polygon and rearrange the individual pieces (while respecting side identifications) to retrieve the original polygon shape for ease of analysis.}

\paragraph{Shearing can be observed in both 2D and 3D spaces. For studying billiards, we will restrict ourselves to 2D shearing maps.}



\section{Periodicity of trajectories}

\paragraph{Identifying sides of a polygon to create a surface in 3D gives us (as we have discussed) a way to represent a trajectory on the polygon in terms of a continuous line, with a single well defined slope in the flat plane. There is another (and essentially similar) way of representing this trajectory, as follows:}

\paragraph{We assume the projectile on the table starts off in a particular direction (say, angle $\theta$ and slope $m = tan\theta$). When it hits a particular side, we simply create a mirror reflection of the polygon about that side (“unfold” the table in the direction of that side), and let the projectile continue on its path in the same direction. Every time it hits a side, we unfold the polygon along that side (Fig 1.9).}

%Fig 1.9

\paragraph{The result is that we get a tiled representation of the flat plane (a graph, of sorts) on which the trajectory is a straight line of slope $m$. We call this the unfolded trajectory.}

\paragraph{We introduce this method because it enables us to easily determine which trajectories in a polygonal table are periodic, and which are not.}

%Fig 1.10

\paragraph{If a trajectory of slope $m$ would be periodic on a particular polygon table, then it means that on the unfolded trajectory, it will pass through the same point on a different polygon as it started from in the first polygon (i.e, it joins corresponding points on two separate polygons on the tiled representation) (Fig 1.10). Conversely, if the trajectory passes through a point on the $n$th polygon that corresponds to the starting position of the first polygon, then the trajectory is periodic. Thus, we know that passing through the same position as the starting position on the $n$th polygon is a sufficient and necessary condition for periodicity. Moreover, the period of the trajectory is $n$.}

\paragraph{To completely describe such trajectories, we will first create a special periodic trajectory, and use it to create more general ones.}

\paragraph{Consider a polygonal billiard table, and the unfolded tiled representation of the table. Consider a trajectory that starts at the center of the first polygon, and passes through the center of the $n$th polygon that it crosses. Clearly, it is a periodic trajectory of period $n$. Let its slope be $m$.}

Fig 1.11

\paragraph{We can create a reconstruction of this trajectory by first creating an ordered sequence of all the polygons that the trajectory passes through, and then joining the centers of the adjacent polygons in the sequence (An illustration of this is shown in (Fig 1.11)). This is a vector decomposition of the trajectory, where each of the constituent vectors are of equal length, and their directions correspond to the normal to each of the sides of the polygon in question. This set of vectors for each polygon can thus be used to construct a center to center trajectory. Call this set $E$.}

\paragraph{We can generalize this by noting that a center to center trajectory can be translated in such a way that it connects any two corresponding points of the $1$st and $n$th polygon, with the same vector decomposition as before (since all vectors stay constant under a translation), as well as the same slope (Fig 1.12). Since we have already seen that the general periodic trajectory always connects corresponding points of two polygons in the tiled representation, we can see that the vectors used to construct a center to center trajectory of slope $m$ can be used to construct any periodic trajectory of slope $m$ (i.e, a trajectory of slope $m$ with any starting point, or simply, the general trajectory of slope $m$).}

%Fig 1.12

\paragraph{There is hence a very powerful relationship between the nature of the polygon, and the periodicity conditions of a trajectory on it. This can be encapsulated in the following theorem:}

\paragraph{$THEOREM$- For a given polygon $P$, let $E_P = {e_1,...e_k}$ be the set of the normal vectors to the sides of the polygon (all of the same length). Then any periodic trajectory on the polygon can be constructed by an integer combination of the vectors in $E_P$. Moreover, the slope for a periodic trajectory is necessarily given by}






\end{document}

