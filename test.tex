\documentclass{article}
\title{Billiards}
\begin{document}
\maketitle

%add a chapter
\section{The rules of billiard dynamics}
\paragraph{Consider a simple polygonal billiard table, and let a projectile be free to move inside it, similar to a cue ball (Fig 1.1). We assume that all collisions with the sides of the square are completely elastic (i.e., angle of incidence = angle of reflection, and no loss of speed occurs after the collision). We further assume that any collision with a vertex will prevent any further motion (one can imagine that the projectile gets “stuck” at the vertex).}

%Fig 1.1

\paragraph{Note- For reasons that will be clear later, we restrict our discussion to tables in the shape of polygons with an even no of sides, with opposite sides being parallel.}
 
\paragraph{With these rules, we may form possible trajectories inside the table. And these trajectories may be studied using various geometrical manipulations. One powerful method that will be the backbone of our analysis is the idea of identifying edges of a polygon- Given a table of a particular polygon shape, each side can be joined (or “identified”) to its opposite parallel side to create a curved surface (Fig 1.2). This is useful, because it allows us to convert any trajectory into a straight line on that curved surface (Fig 1.3)[DJ3] , allowing us to use the geometrical properties of these surfaces to extract information about the trajectories. Note that in the 2D representation of an identified polygon surface, a projectile approaching a side behaves as if it goes into that side, and comes out its identified side (rather like a Pac-man screen)}

\paragraph{Note- It is clear now why we restrict ourselves to polygonal tables with even number of sides- only these shapes can be identified to form a surface.}

%Fig 1.2
%Fig 1.3

\section{Basic terminology and characterization of trajectories}
\paragraph{Identifying a polygon into a surface results, as one may note, into a continuous representation of the trajectory with a single constant slope at all points. Thus, a trajectory can be uniquely identified by this slope. We will refer to it as \textit{m}}

\paragraph{Given that a projectile moves with slope m then, we would like to be able to keep track of its motion within the polygon. Thus, a method of “bookkeeping” of the projectile’s motion is required to characterize trajectories. One way to do so is by keeping track of which sides the projectile goes through. This is made easy by side identification, since it allows us to treat two identified sides as essentially the same entity.}

\paragraph{We can thus label identified sides uniquely, and create a sequence of all the identified sides that the projectile goes through (Fig 1.4 ). If we write this sequence like a word, then we attain what is known as a cutting sequence. Fig 4 shows some examples}

%Fig 1.4

\paragraph{Describing a trajectory as a cutting sequence, we may try to ask some explicit questions about the trajectory, such as}

\begin{itemize}
\item Given a polygon (and hence the number of labelling letters), which sequences of these letters form a valid cutting sequence?

\item Given a valid cutting sequence, is there any way to use them to determine which trajectories are periodic and which are not?

\item Do cutting sequences uniquely represent a single trajectory (does a unique mapping exist between m and a cutting sequence?), or a group of trajectories? And if so, what kind of precise relationship exists between cutting sequences and trajectories?

\end{itemize}

\paragraph{With these basic rules and tools, we now delve into an analysis of these polygon tables geometrically}

\end{document}

